\section{Prova de conceito (POC)}
Para validar a aplicação dos LLMs na automação da avaliação de atendimento ao cliente, foi realizada uma prova de conceito utilizando a API da OpenAI, especificamente o modelo gpt-3.5-turbo. O objetivo deste teste foi verificar, na prática, a viabilidade do projeto.

\subsection{Critérios de avaliação de sucesso}
Os resultados das avaliações devem representar, de forma consistente, a realidade do atendimento realizado.

\subsection{Coleta e preparação de dados}
Inicialmente, foram coletados dados textuais dos logs de chat, que abrangem as comunicações entre clientes e atendentes. Os nomes dos participantes da conversa foram alterados para ``CLIENTE'' e ``ESPECIALISTA DE SUPORTE''. Todos os links foram anonimizados como [link anonimizado] e números específicos de ticket como ``Ticket Nº[anônimo]'' para preservar a privacidade.

A seguir, apresenta-se um exemplo de conversa com cliente utilizada na POC:

\begin{tcolorbox}[title=Primeira conversa usada na POC, label={box:chat-poc}]
-Bate-papo iniciado: Thursday, December 07, 2023, 13:11:36 (-0300)\\
( 6s ) Assistente Virtual: Olá, eu sou a assistente virtual da Efí. \emoji{grinning} Como posso te ajudar hoje?\\
( 8s ) Assistente Virtual: Escolha uma das opções:\\
( 10s ) Assistente Virtual: {ChatWindowButton\\
Abertura de Conta,Acesso à Conta,Autenticador,Assinatura Eletrônica,Maquininha,Nova Versão Aplicativo,Outras Dúvidas}\\
( 20s ) Usuário: Acesso à Conta\\
( 23s ) Assistente Virtual: Selecione uma das opções que irá te auxiliar na sua dúvida:\\
( 25s ) Assistente Virtual: {ChatWindowButton\\
por Atualização Cadastral,Bloqueio de Acesso à Conta,Esqueci Minha Senha}\\
( 30s ) Usuário: Bloqueio de Acesso à Conta\\
( 33s ) Assistente Virtual: Devido à necessidade de confirmação de alguns dados, este atendimento deve ser realizado por meio de um contato telefônico.\\
( 35s ) Assistente Virtual: Pedimos que entre em contato conosco por meio dos números:\\
Ligações originadas de telefone fixo:\\
• 0800 941 2343\\
• ( DDD da operadora + DDD da sua cidade) 4000-1234\\
( 38s ) Assistente Virtual: Capitais e regiões metropolitanas originadas de celular:\\
• 4000 1234\\
• (11) 2394 2208\\
Caso não consiga entrar em contato conosco por estes canais de atendimento, sugerimos que nos encaminhe um Ticket de Suporte.\\
( 40s ) Assistente Virtual: As informações foram úteis?\\
( 43s ) Assistente Virtual: {ChatWindowButton\\
\emoji{smiley},Não \emoji{confused}}\\
( 3m 43s ) Usuário: Não \emoji{confused}\\
( 3m 46s ) Assistente Virtual: O que você deseja fazer?\\
( 3m 48s ) Assistente Virtual: {ChatWindowButton\\
principal,Falar com Atendente}\\
( 3m 51s ) Usuário: Falar com Atendente\\
( 3m 56s ) Assistente Virtual: Aguarde enquanto busco um especialista pra falar com você.\\
Você sabia que a Efí não solicita nenhum dado de acesso à conta ou para realizar uma transação, seja por redes sociais, WhatsApp, telefone ou outro canal?\\
Bate-papo iniciado: Thursday, December 07, 2023, 13:15:32 (-0300)\\
Origem do bate-papo: Fila Suporte\\
Bate-papo transferido de Assistente Virtual para Especialista de Suporte\\
( 4m 5s ) Especialista de Suporte: Olá, cliente.\\
Meu nome é [Especialista de Suporte], estou aqui para ajudar você!\\
( 4m 12s ) Usuário: Oi\\
( 4m 18s ) Usuário: Boa tarde\\
( 4m 43s ) Usuário: Recebi um e-mail informando o bloqueio da minha conta e mandando entrar em contato\\
( 4m 50s ) Usuário: Qual o motivo?\\
( 5m 33s ) Especialista de Suporte: Identifiquei que a sua conta está sob um procedimento de análise, para sua segurança. Esse procedimento ocorre sob o [link anonimizado], no qual você precisa interagir com as informações solicitadas. Acesse ao menu "Ajudas e Sugestões > Estou com um problema > Acessar meus Tickets > [link anonimizado]" e aguarde o retorno, por gentileza.\\
( 6m 13s ) Usuário: Análise se ela foi aprovada\\
( 8m 31s ) Usuário: ?\\
( 9m 18s ) Especialista de Suporte: Um momento.\\
( 10m 42s ) Usuário: Já liguei e a ligação não completa\\
( 13m 26s ) Especialista de Suporte: Nesse caso, peço que interaja no Ticket Nº[anônimo], por gentileza. Acesse ao menu "Ajudas e Sugestões > Estou com um problema > Acessar meus Tickets > Ticket Nº[anônimo]", por favor.\\
\end{tcolorbox}

\newpage
\subsection{Implementação dos testes}
Utilizando o modelo gpt-3.5-turbo, as conversas de chat foram submetidas à análise. O modelo foi configurado para interpretar as interações, avaliar a qualidade do atendimento e gerar insights acionáveis sobre as conversas. A configuração da API incluiu ajustes nos prompts para garantir que as respostas fossem consistentes e úteis.
A seguir, apresenta-se o prompt da prova de conceito:

\begin{lstlisting}[language=Python, caption={Prompt da prova de conceito},style=Python, label={lst:prompt-poc}]
prompt = f"""
Analise a seguinte conversa com um cliente em nossos canais de atendimento e forneça as informações solicitadas em JSON:

{conversa}

Forneça uma análise detalhada da conversa incluindo:
- tags: Lista de palavras-chave do atendimento.
- sentimento: Cliente está feliz, chateado ou algo assim?
- problema: Qual foi o problema do cliente?
- resolucao: Teve resolução? Qual?
- sugestao: Sugestão de melhoria para nosso atendimento.
- produtos: Quais produtos o cliente teve problema?
- resumo: Resumo do atendimento, interpretando os dados acima também.
"""
\end{lstlisting}

\subsection{Análise dos resultados}

A qualidade das análises automáticas demonstrou-se consistente com a realidade das conversas, indicando que o modelo foi capaz de capturar nuances e contextos importantes nas interações. As principais métricas avaliadas foram:

\begin{itemize}
    \item Identificação de pontos críticos nas conversas;
    \item Coerência na interpretação dos sentimentos dos clientes.
\end{itemize}

A seguir, apresenta-se o resultado da avaliação da conversa descrita na seção (\ref{box:chat-poc}) \textit{Coleta e Preparação de Dados}:

\begin{lstlisting}[language=JSON, caption={Resultado da Avaliação da Conversa}, style=customJSON]
ChatCompletionMessage(  
    content='{  
        "tags": ["acesso à conta", "bloqueio de conta", "ticket de suporte"],  
        "sentimento do cliente": "chateado",
        "problema": "Recebeu um e-mail informando o bloqueio da conta",
        "resolucao": "Encaminhado para acessar o Ticket Nº[anônimo] e aguardar retorno",
        "sugestao": "Melhorar a comunicação sobre procedimentos de bloqueio de conta",
        "produtos": ["conta"],
        "resumo": "O cliente [CLIENTE] estava chateado pois recebeu um e-mail informando o bloqueio da conta. O atendente [ESPECIALISTA DE SUPORTE] identificou que a conta estava sob análise e encaminhou para acessar um ticket de suporte. O cliente reclamou que a ligação não completou e o atendente orientou a interagir no ticket. Sugestão de melhoria é melhorar a comunicação sobre procedimentos de bloqueio de conta."
    }', 
    role='assistant', 
    function_call=None, 
    tool_calls=None
)
\end{lstlisting}

Durante a prova de conceito, foram identificados desafios, como a necessidade de refinamento dos \textit{prompts} para o cenário de produção e a gestão de grandes volumes de dados. A prova de conceito demonstrou o potencial dos LLMs na automação cognitiva de processos de avaliação de atendimento ao cliente.
