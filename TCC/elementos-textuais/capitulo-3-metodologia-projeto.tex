\section{Implementação do projeto}

\subsection{Escopo de canais para a avaliação}

\begin{itemize}
    \item Chat web
    \item Chat app
    \item Chat WhatsApp
    \item Chat Discord
    \item E-mail
    \item Chamados de atendimento em ambiente de conta logada
    \item Notas internas dos agentes de atendimento
\end{itemize}

O processo de atendimento em que se aplicou o estudo é omnichannel, permitindo que o cliente transite entre diferentes canais durante uma mesma interação. Um desafio inicial foi o tamanho excessivo do prompt de entrada, que incluía as conversas e as notas dos agentes, tornando-o incompatível com as limitações da API da OpenAI. Para mitigar esse problema, cada conversa por chat foi pré-analisada utilizando o mesmo prompt, e o resultado dessa análise foi incorporado ao prompt de avaliação final de um chamado completo, juntamente com os dados dos demais canais.

\newpage
\subsection{Visão geral do processo de atendimento omnichannel e consumo da API}

A seguir, apresenta-se um fluxograma que ilustra o processo geral em que este estudo foi aplicado. Os blocos destacados em laranja representam as áreas de foco deste estudo.

\begin{figure}[h]
    \centering
    \includegraphics[width=\textwidth]{figuras/Processo de atendimento.png}
    \caption[Fluxograma do processo de atendimento]{Visão geral do processo de atendimento omnichannel e consumo da API.}
    \label{fig:fluxograma-atendimento}
    \legend{Fonte: Efí, 2024.}
\end{figure}

\newpage
\subsection{Prompt construído}

O prompt elaborado demonstra boas práticas de prompting ao combinar clareza, especificidade, organização e controle rigoroso do escopo das respostas. O uso de exemplos específicos e a limitação das opções permitidas minimizam ambiguidades, enquanto a formatação padronizada das respostas facilita o processamento subsequente. 

Essa abordagem é coerente com as técnicas discutidas por Qiao et al. (2023), que destacam a eficácia da divisão do raciocínio em múltiplos estágios para tarefas complexas, além da importância de fornecer instruções detalhadas para minimizar ambiguidades e melhorar a precisão dos resultados. 

A seguir, apresenta-se a versão mais recente do prompt no momento da escrita deste estudo.

\begin{tcolorbox}[title=Prompt final, label={box:prompt-final}]
Você é um especialista em atendimento de suporte técnico na Instituição de Pagamento Efí Bank. Você receberá um JSON com o conteúdo de todo o atendimento realizado, podendo conter conversas de chat, whatsapp, email e outros.\\
--------------\\
IMPORTANTE:\\
Tenha atenção redobrada nas interpretações. Não tire conclusões precipitadas.\\
Ao mencionar pessoas ou equipes sempre especifique de qual empresa são, por exemplo: "a documentação será enviada pelo financeiro do cliente" ao invés de "a documentação será enviada pelo financeiro". se a equipe for da efí, mencione também, por exemplo, "equipe de suporte da Efí"\\
Você receberá datas no formato "2024-06-07T09:40:03.000Z" mas para as respostas que envolverem datas, adote o formato dd/mm/aaaa.\\
--------------\\
LISTA DE SENTIMENTOS que usaremos: satisfeito (\emoji{grinning}), neutro (\emoji{neutral-face}), esperançoso (\emoji{pray}), curioso (\emoji{thinking}), preocupado (\emoji{worried}), frustrado (\emoji{persevere}), confuso (\emoji{confused}), desesperado (\emoji{weary}), indiferente (\emoji{expressionless}), irritado (\emoji{angry}), triste (\emoji{cry}), ansioso (\emoji{anxious-face-with-sweat})\\
Analise a seguinte conversa com um cliente em nossos canais de atendimento e forneça as informações solicitadas em JSON:\\
Forneça uma análise detalhada da conversa incluindo:\\
- Motivo: Principal motivo pelo qual o cliente entrou em contato. Por eemplo: Dúvidas, Problemas, Solicitações, Sugestões, Reclamações, Acompanhamento Interno...\\
- Assunto: Crie um assunto que ilustre da melhor forma possível este atendimento.\\
- Tags: Lista de principais palavras-chave do atendimento (Evite tags não relevantes como "atendimento ao cliente" ou "ticket de suporte"). Para reduzir a redundância entre esta e futuras avaliações, viabilizar normalização em relatórios e manter a informação relevante, aqui estão as tags principais: Cartão de Crédito, Cartão de Débito, Cartão Pré-Pago, Cartão Virtual, Emissão de Cartão, Bloqueio de Cartão, Cancelamento de Cartão, Perda/Roubo de Cartão, Segunda Via de Cartão, Transferência Pix, Cobrança Pix, Chave Pix, QR Code Pix, Erro Pix, Bloqueio de Pix, Devolução Pix, Abertura de Conta, Encerramento de Conta, Bloqueio de Conta, Conta Digital, Conta Física, Conta Jurídica, Conta Poupança, Redefinição de Senha, Troca de Senha, Bloqueio de Senha, Senha Eletrônica, Envio de Documentos, Documentos Pessoais, Documentos Empresariais, Documentação Pendente, Validação de Documentos, Contestação de Transação, Transação Não Reconhecida, Transação Bloqueada, Erro de Transação, Pagamento Pix, Pagamento Boleto, Pagamento de Contas, Pagamento Não Identificado, Pagamento Duplicado, Erro no Sistema, Erro no Aplicativo, Erro de Autenticação, Erro de Pagamento, Erro de Cadastro, Autenticação de Dispositivo, Autenticação em Dois Fatores, Validação de Identidade, Validação de Selfie, Autenticação Facial, Emissão de Boleto, Pagamento de Boleto, Boleto Não Registrado, Boleto Duplicado, Boleto Não Pago, Limites(para lidar com solicitações de aumento ou redução de limite), Estorno de Pagamento (para lidar com devoluções e reembolsos), Acesso ao Aplicativo (para problemas específicos de login ao aplicativo), Acesso Web (para problemas específicos de login a conta pelo navegador), Atualização de Cadastro (para mudanças em informações pessoais ou empresariais). Se não encontrar uma tag, pode criar novas tags.\\
- Problema: Qual foi o problema do cliente?\\
- Resolucao: Teve resolução? Qual?\\
- Produtos: Retorne os produtos que o cliente teve problema. Para reduzir a redundância entre esta e futuras avaliações, viabilizar normalização em relatórios e manter a informação relevante, aqui estão os principais produtos: (Cartão, Cartão de débito, Cartão de crédito, Cartão pré-pago, Maquininha, Pix, Boletos, Carnês, APIs, Gestão de cobranças, Links de pagamento, Assinaturas, Checkout transparente, Links de pagamento, Split de pagamento, TED, Efí Invest, API Boletos, API Pix, API Open Finance). Se não encontrar um produto na lista, crie novos produtos.\\
- Resumo: Resumo do atendimento. Inclua detalhes como datas, números, etc. Mencione os nomes dos envolvidos. VERIFIQUE AS DATAS ANTES DE ESCREVER PARA manter em ordem cronológica. IGNORE O PAR METRO "ActorName". Você não deve, em hipótese alguma, usar termos como "o sistema respondeu" e sim o nome da pessoa que assina a mensagem. Não omita detalhes importantes.\\
- Sentimento: Responda em uma palavra qual o sentimento do cliente. Selecione da LISTA DE SENTIMENTOS, informada acima. NÃO TIRE CONCLUSÕES PRECIPITADAS. Na dúvida, responda apenas "indefinido".\\
- SentimentoEmoji: Emoji com o sentimento do cliente. Selecione da LISTA DE SENTIMENTOS, informada acima. NÃO TIRE CONCLUSÕES PRECIPITADAS. Na dúvida, responda apenas "•".\\
- SentimentoFinal: Responda em uma palavra qual o sentimento do cliente ao final do atendimento. Selecione da LISTA DE SENTIMENTOS, informada acima. Só responda se for possível inferir claramente a partir das últimas respostas do cliente. NÃO TIRE CONCLUSÕES PRECIPITADAS. Na dúvida, responda apenas "indefinido".\\
- SentimentoFinalEmoji: Emoji com o sentimento do cliente ao final do atendimento. Selecione da LISTA DE SENTIMENTOS, informada acima. Só responda se for possível inferir claramente a partir das últimas respostas do cliente. SÓ RESPONDA UM SENTIMENTO SE ESTIVER MUITO CLARO. SEM DÚVIDAS. NÃO TIRE CONCLUSÕES PRECIPITADAS. Na dúvida, responda apenas "•".\\
- Sugestao: Essa demanda do cliente poderia ser resolvida por um chatbot? Se sim, como?\\
Verifique cuidadosamente suas respostas antes de concluir para assegurar que todas as informações fornecidas se alinhem estritamente com os requisitos. Além disso, atente-se às datas das mensagens trocadas. Sua análise detalhada ajudará a melhorar nosso atendimento ao cliente dentro dos parâmetros estabelecidos.\\
Retorne um JSON em uma linha sem espaços em branco e com todas as chaves começando com a primeira letra maiúscula e o restante minúsculo, no valor das chaves mantenha o padrão: primeira letra da primeira palavra como maiúscula e restante minúsculo. Não crie uma chave em volta das informações solicitadas.\\
\end{tcolorbox}

\subsection{Discussão sobre o prompt}

\subsubsection{Clareza e especificação do contexto}
O prompt é introduzido com uma clara definição de contexto: ``Você é um especialista em atendimento de suporte técnico em uma fintech''. Essa instrução estabelece uma expectativa de expertise no domínio específico, preparando o modelo para interpretar os dados de maneira contextualizada. A continuação, ``Você receberá um JSON com o conteúdo de todo o atendimento realizado, podendo conter conversas de chat, whatsapp, email e outros'', delimita o escopo dos dados a serem analisados, garantindo que o modelo compreenda o formato e a abrangência das informações fornecidas.

\subsubsection{Estrutura e organização}
O prompt é organizado, detalhando as informações específicas a serem extraídas de cada interação. Por exemplo, para o campo ``Motivo", o prompt solicita que o modelo identifique o ``Principal motivo pelo qual o cliente entrou em contato", oferecendo sugestões como ``Dúvidas, Problemas, Solicitações, Sugestões, Reclamações, Acompanhamento Interno". Essa abordagem guiada minimiza interpretações errôneas e direciona o modelo a respostas dentro do escopo esperado.

A seção de \textit{Tags} é abrangente, fornecendo uma lista extensa de termos como ``Cartão de Crédito, Cartão de Débito, Cartão Pré-Pago, Bloqueio de Cartão, etc". Essa especificidade não apenas facilita a categorização das interações, mas também permite a criação de novas \textit{tags}, caso as pré-definidas não sejam adequadas: ``Se não encontrar uma tag, pode criar novas tags''.

A lista de \textit{tags} no \textit{prompt} foi levantada a partir de uma semana de análises de atendimentos utilizando uma versão mais aberta para as \textit{tags}.

\subsubsection{Especificação do formato de saída}
O formato de saída é especificado para assegurar a uniformidade dos dados: ``Retorne um JSON em uma linha sem espaços em branco e com todas as chaves começando com a primeira letra maiúscula e o restante minúsculo''. Além disso, o \textit{prompt} instrui que ``no valor das chaves, mantenha o padrão: primeira letra da primeira palavra como maiúscula e restante minúsculo''. Essas diretrizes detalhadas são fundamentais para garantir a consistência no processamento automatizado e na análise posterior dos dados.

\subsubsection{Normalização de respostas e limitação do escopo}
Os campos ``Motivo'', ``Tags'', ``Sentimento'', ``SentimentoEmoji'', ``SentimentoFinal'', ``SentimentoFinalEmoji'' e ``Produtos'' foram elaborados para garantir a normalização das respostas, o que é fundamental para a construção de relatórios quantitativos. A normalização desses campos permite a padronização das respostas, facilitando a agregação de dados e a realização de análises estatísticas.

Para os campos ``Produtos'' e ``Tags'', o prompt instrui o modelo a priorizar a lista de referência, mas permite a criação de novos itens, se necessário. Para ``Produtos'': ``Se não encontrar um produto na lista, crie novos produtos''. E para ``Tags'': ``Se não encontrar uma tag, pode criar novas tags''. A criação de novas tags e produtos é permitida apenas quando os itens listados não se aplicam, assegurando que o modelo forneça respostas relevantes. Isso possibilita a identificação de novos produtos que os clientes podem estar demandando e de temas ainda não mapeados na lista de tags.

Para a categorização de sentimentos, é fornecida uma ``LISTA DE SENTIMENTOS'', com exemplos como satisfeito (\emoji{grinning}), neutro (\emoji{neutral-face}), esperançoso (\emoji{pray}), curioso (\emoji{thinking}), preocupado (\emoji{worried}), frustrado (\emoji{persevere}), confuso (\emoji{confused}), desesperado (\emoji{weary}), indiferente (\emoji{expressionless}), irritado (\emoji{angry}), triste (\emoji{cry}), ansioso (\emoji{anxious-face-with-sweat}). Essa lista é essencial para padronizar as respostas, evitando ambiguidades e promovendo consistência.

\subsubsection{Verificação de respostas}
A conclusão do \textit{prompt} inclui uma instrução para verificação das respostas: ``Verifique cuidadosamente suas respostas antes de concluir para assegurar que todas as informações fornecidas se alinhem estritamente com os requisitos''. Esta orientação ajudou a garantir que as respostas estejam corretas e aderentes às especificações fornecidas, mantendo a integridade dos dados. A atenção às ``datas das mensagens trocadas'' é enfatizada para assegurar a precisão temporal na análise.

\subsubsection{Campos qualitativos}
Além dos campos quantitativos normalizados, o prompt também inclui campos qualitativos que permitem uma análise contextualizada dos atendimentos ao cliente. Esses campos incluem ``Assunto'', ``Problema'', ``Sugestao'', e ``Resumo'', e são projetados para capturar nuances e detalhes que não podem ser facilmente quantificados.

\subsubsection{Assunto}
O campo ``Assunto'' solicita ao modelo que crie um título que sintetize o tema principal do atendimento. A instrução é: ``Crie um assunto que ilustre da melhor forma possível este atendimento''. Este campo foi projetado pois nem sempre um assunto de chamado de atendimento é totalmente compatível com o conteúdo, ajudando a identificar o foco principal de cada caso.

\subsubsection{Sugestão}
O campo ``Sugestão'' é projeto para utilização em melhoria contínua, solicitando que o modelo forneça sugestões de melhorias tanto para o analista de atendimento quanto para os processos internos da empresa. A instrução detalha: ``Por favor, forneça sugestões de melhoria para o analista de atendimento e nossos processos. O que o analista poderia ter feito para melhorar nosso atendimento? Quais melhorias podemos implementar em nossos processos?'' Além disso, o prompt pede sugestões de marketing e produto, destacando oportunidades para melhorar a experiência do cliente e explorar ``gatilhos reptilianos'' — elementos psicológicos que podem influenciar o comportamento do consumidor. Esse campo é essencial para identificar áreas de melhoria e inovação, bem como para reforçar práticas bem-sucedidas.

\subsubsection{Resumo}
Por fim, o campo ``Resumo'' solicita uma síntese concisa de todo o atendimento, incluindo informações como datas, números e nomes relevantes: ``Resumo conciso do atendimento, interpretando todos os dados. Inclua detalhes como datas, números, nomes, etc''. Este campo serve para condensar a interação em um formato resumido, facilitando a revisão e análise de casos específicos.

\subsection{Preparo dos dados}

\subsubsection{Recuperação dos dados}
Como o atendimento é omnichannel e focado em canais de texto, foi necessário coletar dados de diferentes fontes, como WhatsApp, App, Chat web, Discord, e-mail e respostas no ambiente logado das contas dos clientes. Isso introduziu um risco de consistência de dados que impactou a qualidade dos resumos, pois a cronologia das conversas pode estar fragmentada entre essas bases de dados. Como este é um estudo de prova de conceito (PoC), reorganizar esses dados de forma cronológica não foi viável para a versão utilizada no estudo. No entanto, esse problema foi mitigado incluindo instruções diretamente no prompt, como mostrado a seguir:

\begin{tcolorbox}
VERIFIQUE AS DATAS ANTES DE ESCREVER PARA manter em ordem cronológica.
\end{tcolorbox}

\subsubsection{Limpeza dos dados}
Remoção de caracteres não relevantes, como HTML, espaços em branco e quebras de linha duplicadas. Os dados de um atendimento (chats, e-mails, notas internas dos agentes) foram organizados em JSON.

A seguir, apresenta-se uma amostra de dados de um atendimento completo. Neste exemplo, as quatro primeiras mensagens resultam de um resumo gerado a partir de uma conversa no WhatsApp, utilizando o mesmo prompt. No JSON apresentado, todas as informações que poderiam identificar pessoas ou links foram substituídas por ``[Cliente]'', ``[Agente de atendimento]'' e ``[Link]''.

\begin{lstlisting}[language=JSON, caption={Amostra de dataset}, style=customJSON]
{
  "messages": [
    {
      "content": {
        "messagingSessionSummaries": [
          {
            "Summary": "Em 31 de julho de 2024, a cliente [Cliente] entrou em contato via WhatsApp relatando problemas para acessar sua conta e a necessidade de atualizar o endereço para a entrega de um cartão. A atendente [Agente de atendimento] orientou a cliente a redefinir a senha através de um e-mail enviado e a atualizar o endereço via ticket de suporte. A cliente foi instruída a seguir os procedimentos e ficou esperançosa com a resolução.",
            "StartTime": "2024-07-31T17:54:29.000Z",
            "EndTime": "2024-07-31T18:38:03.000Z"
          },
          {
            "Summary": "Em 1 de agosto de 2024, a cliente [Cliente] entrou em contato via WhatsApp relatando que não conseguia acessar sua conta após ter seu celular roubado e perder o número de telefone cadastrado. O analista [Agente de atendimento] verificou a situação e informou que seria necessário um bloqueio temporário da conta e a validação de identidade para resolver o problema. A cliente forneceu os dados solicitados e foi informada que um analista entraria em contato por telefone para finalizar o processo. A cliente terminou o atendimento esperançosa de que o problema seria resolvido em breve.",
            "StartTime": "2024-08-01T18:46:03.000Z",
            "EndTime": "2024-08-01T19:37:52.000Z"
          },
          {
            "Summary": "Em 02 de agosto de 2024, a cliente [Cliente] entrou em contato via WhatsApp relatando dificuldades para acessar sua conta, pois o código de verificação estava sendo enviado para um número de telefone antigo. Após várias tentativas e frustrações, o analista [Agente de atendimento] reenviou o e-mail com o procedimento correto para alterar o número de telefone e a senha. A cliente foi orientada a seguir os passos e aguardar a confirmação por e-mail. O atendimento foi finalizado com a cliente esperançosa de que o problema seria resolvido.",
            "StartTime": "2024-08-02T18:15:08.000Z",
            "EndTime": "2024-08-02T18:52:53.000Z"
          },
          {
            "Summary": "Em 05/08/2024, a cliente [Cliente] entrou em contato via WhatsApp relatando problemas para acessar o aplicativo devido à troca de número de telefone, o que impedia a redefinição de senha. A analista [Agente de atendimento] orientou a cliente a seguir os passos do ticket de suporte já aberto e aguardar o retorno do time responsável após o envio dos documentos necessários. A cliente expressou frustração com a dificuldade de acesso, mas a conversa terminou de forma neutra.",
            "StartTime": "2024-08-05T17:17:59.000Z",
            "EndTime": "2024-08-05T17:55:12.000Z"
          }
        ],
        "messagesIntranet": null,
        "liveChatTranscriptSummaries": [],
        "emailMessages": [
          {
            "Message": "Olá, [Cliente]. Estamos encaminhando abaixo o procedimento a ser realizado para a alteração de seus dados cadastrais. É muito importante que você nos informe, respondendo a esse mesmo ticket, quando finalizar o envio dos documentos solicitados, para que possamos seguir com as alterações. Após recebidos os documentos através do link e a informação de que o processo foi concluído, o retorno será dado em até 01 (um) dia útil. Para iniciar o processo de Alteração dos Dados Cadastrais da sua Conta Digital, pedimos por gentileza, para acessar o link abaixo, através do seu smartphone: Link: [Link] Tenha em mãos o seu RG ou a sua CNH, para que possamos validar a sua identidade e, ao final do processo, informe os dados cadastrais que deseja alterar. Lembramos que a Efí não solicita nenhum dado de acesso à sua conta nem a realização de transação financeira por redes sociais, WhatsApp, telefone, e-mail ou outros canais. Conte com a gente! Atenciosamente, [Agente de atendimento]",
            "EntryDate": "2024-08-02T18:31:05.000Z",
            "ActorName": "System"
          },
          {
            "Message": "Olá, [Cliente]. Confirmo o recebimento da sua comunicação e informo que o setor responsável foi acionado. Por favor, aguarde o retorno do nosso setor, que será conduzido por meio deste ticket. Permaneço à disposição! Conte com a gente! Atenciosamente, [Agente de atendimento]",
            "EntryDate": "2024-08-01T20:09:14.000Z",
            "ActorName": "System"
          }
        ]
      }
    }
  ]
}
\end{lstlisting}

\subsection{Hiperparâmetro temperatura}

O hiperparâmetro ``temperatura''¹ nas APIs da OpenAI desempenha um papel crucial na determinação da criatividade e variação nas respostas geradas pelos modelos de linguagem. Esse parâmetro controla a aleatoriedade das previsões: valores mais baixos (próximos de 0) fazem com que o modelo produza respostas mais determinísticas e previsíveis, enquanto valores mais altos (próximos de 1) permitem maior diversidade nas saídas, incentivando respostas mais criativas.

Para diferentes tipos de tarefas, recomenda-se ajustar a temperatura conforme a necessidade específica. Por exemplo, para tarefas de criação de conteúdo ou brainstorming, utilizam-se frequentemente valores de temperatura mais altos, como 0,7 ou 0,8, para promover a geração de ideias mais variadas e criativas. Em contraste, para tarefas que exigem maior precisão e consistência, como a avaliação de atendimentos ao cliente, é preferível utilizar valores baixos de temperatura para garantir que as respostas sejam mais objetivas e alinhadas com a realidade do contexto.

\begin{quote}
\small ¹ OPENAI. API Reference - Temperature. Disponível em: \url{https://platform.openai.com/docs/guides/text-generation/how-should-i-set-the-temperature-parameter}. Acesso em: 09 ago. 2024.
\end{quote}

Neste estudo, onde se busca um comportamento mais ``conservador'' e aderente aos dados, experimentaram-se diferentes configurações de temperatura e, após análise, optou-se por utilizar o valor de 0,1. Este ajuste foi escolhido para assegurar que o modelo operasse de maneira consistente e com foco na precisão, minimizando desvios criativos que poderiam comprometer a avaliação dos atendimentos.

\subsection{Automação}
O processo de avaliação é disparado automaticamente cada vez que um chamado é encerrado. Se o cliente ou os agentes reabrem o chamado, atualizam e encerram novamente, a avaliação é atualizada, conforme ilustrado na Figura \ref{fig:fluxograma-atendimento} - Visão geral do processo de atendimento omnichannel e consumo da API.

\subsection{gpt-4o-mini}
Foram realizados testes utilizando o modelo gpt-4o-mini, que é mais rápido e apresenta um custo 97\% menor em comparação ao gpt-4o. Mais detalhes no Capítulo 4 - Resultados.