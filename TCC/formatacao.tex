% ---
% Arquivo com a formatação do TCC.
% Este arquivo não deve ser modificado.
% ---
%\usepackage[utf8]{inputenc}
\usepackage[T1]{fontenc}
\usepackage{amsmath}
\usepackage{amssymb,amsfonts,textcomp}
\usepackage{xcolor}
\usepackage{tcolorbox}
\usepackage{array}
\usepackage{supertabular}
\usepackage{lastpage}		  % Usado pela Ficha catalográfica
\usepackage{indentfirst}	  % Indenta o primeiro parágrafo de cada seção.
\usepackage{hhline}
\usepackage{hyperref}
%\usepackage[pdftex]{graphicx}
\usepackage{graphicx}
\graphicspath{ {./figuras/} }
\usepackage{emoji}
\usepackage[multiple]{footmisc}
\usepackage{breakcites}
\tcbuselibrary{breakable}
\usepackage{alltt}
\usepackage{minted}
\usepackage{placeins}
\usepackage{pdfpages}

% retira as mensagens de aviso do pacote Glossaries
\let\printglossary\relax
\let\theglossary\relax
\let\endtheglossary\relax
% coloca Seções em maiúsculas sem negrito no Sumário
\makeatletter
\let\oldcontentsline\contentsline
\def\contentsline#1#2{%
  \expandafter\ifx\csname l@#1\endcsname\l@section
    \expandafter\@firstoftwo
  \else
    \expandafter\@secondoftwo
  \fi
  {%
    \oldcontentsline{#1}{\normalfont\MakeTextUppercase{#2}}%
  }{%
    \oldcontentsline{#1}{#2}%
  }%
}
\makeatother
% ---
% Pacotes glossaries
% ---
\usepackage[subentrycounter,seeautonumberlist,nonumberlist=true]{glossaries}
% para usar o xindy ao invés do makeindex:
%\usepackage[xindy={language=portuguese},subentrycounter,seeautonumberlist,nonumberlist=true]{glossaries}
% ---
% Citações de referências no formato alfabético e negrito
\usepackage[alf, abnt-emphasize=bf]{abntex2cite} 
% Margens definidas em 25 mm para uso como documento em PDF
% Para imprimir use as seguintes margens:
% \usepackage[left=30mm, top=30mm, right=20 mm, bottom=20mm] {geometry}

\usepackage[margin=25 mm]{geometry}
%---
% O arquivo com o nome dos alunos e dos orientadores é lido aqui.
% Atualizar diretamente no arquivo
%---
%%%%%%%%%%%%%%%%%%%%%%%%%%%%%%%%%%%%%%%%%%%%%%%%%%%%%%%%%%%%%
% Definições de Macros utilizadas no TCC - ATUALIZE AQUI
%%%%%%%%%%%%%%%%%%%%%%%%%%%%%%%%%%%%%%%%%%%%%%%%%%%%%%%%%%%%%
\instituicao{INSTITUTO FEDERAL DE MINAS GERAIS - \textit{CAMPUS} OURO PRETO
\par
CURSO ESPECIALIZAÇÃO EM INTELIGÊNCIA ARTIFICIAL}
%\title{Análise do Uso de Inteligências Artificiais na Gestão de Relacionamento com o Cliente\\ Modelo em Latex}
\title{Aplicação de Modelos de Linguagem de Grande Escala (LLM) na Automação Cognitiva: Uma Prova de Conceito em Atendimento ao Cliente}
\autor{Alexandre Fortes Santana}
\orientador{Prof. Frederico Gadelha Guimarães (UFMG)}
\local{Ouro Preto}
\data{2024}
\preambulo{Trabalho de conclusão do curso de Especialização em Inteligência Artificial do Instituto Federal de Minas Gerais - \textit{Campus} Ouro Preto para a obtenção do certificado de conclusão.}

%%%%%%%%%%%%%%%%%%%%%%%%%%%%%%%%%%%%%%%%%%%%%%%%%%%%%%%%%%%
% Corrige a fonte dos capítulos, seções, resumos, etc.
%%%%%%%%%%%%%%%%%%%%%%%%%%%%%%%%%%%%%%%%%%%%%%%%%%%%%%%%%%%

\renewcommand{\ABNTEXchapterfont}{\bfseries \rmfamily}  % Capítulos em Bold e Maiúsculas
\renewcommand{\ABNTEXchapterfontsize}{\normalsize}
\renewcommand{\ABNTEXsectionfont}{\rmfamily}            %  Seções em  Maiúsculas apenas
\renewcommand{\ABNTEXsectionfontsize}{\normalsize}
\renewcommand{\ABNTEXsubsectionfont}{\bfseries}         % Subseções em Bold apenas
\renewcommand{\ABNTEXsubsectionfontsize}{\normalsize}

\usepackage{url16023}  % para retirar < e > da URL nas referências.

%%%%%%%%%%%%%%%%%%%%%%%%%%%%%%%%%%%%%%%%%%%%%%%%%%%%%%%
% Redefine a macro para imprimir a capa 
%%%%%%%%%%%%%%%%%%%%%%%%%%%%%%%%%%%%%%%%%%%%%%%%%%%%%%%
\renewcommand{\imprimircapa}{%
\begin{capa}%
\center
\imprimirinstituicao
\par
\vspace*{1cm}
\imprimirautor
\vfill
\begin{center}
\MakeUppercase{\bfseries\imprimirtitulo}
\end{center}
\vfill
\imprimirlocal
\par
\imprimirdata
\vspace*{1cm}
\end{capa}
}
%%%%%%%%%%%%%%%%%%%%%%%%%%%%%%%%%%%%%%%%%%%%%%%%%%%%%%%
% Redefine a macro para imprimir a folho de rosto
%%%%%%%%%%%%%%%%%%%%%%%%%%%%%%%%%%%%%%%%%%%%%%%%%%%%%%%
\renewcommand{\imprimirfolhaderosto}{%
\begin{capa}%
\center
\par
\vspace*{1cm}
\MakeUppercase{\imprimirautor}
\vfill
\begin{center}
\MakeUppercase{\bfseries\imprimirtitulo}
\end{center}
\vfill
\hspace*{\fill}\parbox[b]{.5\textwidth}{%
        \linespread{1}\selectfont
\imprimirpreambulo
}
\vfill
\flushright
Orientador: \imprimirorientador\\

%Co-orientador: \imprimircoorientador\\
\vfill
\begin{center}
\large\imprimirlocal
\par
\large\imprimirdata
\vspace*{1cm}
\end{center}
\end{capa}
}
%%%%%%%%%%%%%%%%%%%%%%%%%%%%%%%%%%%%%%%%%%%%%%%%%%%%%%%
% Informações do PDF inseridas automaticamente
% Atualizar apenas as palavras-chave se necessário
% Não modificar as cores dos links e referências
%%%%%%%%%%%%%%%%%%%%%%%%%%%%%%%%%%%%%%%%%%%%%%%%%%%%%%%
\makeatletter
\hypersetup{
pdftitle={\@title},
pdfauthor={\@author},
pdfsubject={\imprimirpreambulo},
pdfkeywords={Trabalho de conclusão de curso}{Inteligência artificial}{CRM},
pdfcreator={LaTeX with abnTeX2},
colorlinks=true,
linkcolor=black,
citecolor=black,
urlcolor=black
}
\makeatother
%%%%%%%%%%%%%%%%%%%%%%%%%%%%%%%%%%%%%%%%%%%%%%%%%%%%
% Definição das Linguagens de Programação
%%%%%%%%%%%%%%%%%%%%%%%%%%%%%%%%%%%%%%%%%%%%%%%%%%%%
% ---
% Pacote para formatação de linguagens de programação
% ---
\usepackage{listings}         % Para as linguagens de programação

% ---
% Definição das cores para a formatação de códigos
% ---
\definecolor{dkgreen}{rgb}{0,0.6,0}
\definecolor{gray}{rgb}{0.5,0.5,0.5}
\definecolor{purple}{rgb}{0.8,0,0.3}
\definecolor{orange}{rgb}{1,0.4,0}
\definecolor{lightlightgray}{rgb}{.95,.95,.95}
\definecolor{lightgray}{rgb}{.9,.9,.9}
\definecolor{lightgray2}{rgb}{.85,.85,.85}
\definecolor{darkgray}{rgb}{.4,.4,.4}

% ---
% Comando personalizado para inserir listagens de código
% ---
\newcommand{\includecode}[4][C]{\mbox{\lstinputlisting[caption=#2, label=#3, escapechar=, style=custom#1]{#4}}}

% ---
% Configurações gerais de formatação para listagens de código
% ---
\lstset{numberbychapter=false, framexleftmargin=5mm, frame=shadowbox, rulesepcolor=\color{gray}}

% ---
% Configuração de títulos e legendas para listagens de código
% ---
\renewcommand{\lstlistingname}{Código}                  % Nome para os códigos no texto
\renewcommand{\lstlistlistingname}{Lista de \lstlistingname s}
\makeatletter                                          % Configura a linha da lista de códigos
\renewcommand\l@lstlisting[2]{{\normalfont\@dottedtocline{1}{1.5em}{2em}{Código~#1}{#2}}}
\makeatother

% ---
% Leitura de configurações específicas para formatação de linguagens de programação
% ---
\input{codigos/linguagens}

% Caixas

\usepackage{setspace} % Controle de espaçamento entre linhas
\tcbset{
    colframe=darkgray,     % Cor da borda, um cinza escuro
    fonttitle=\bfseries,   % Título em negrito
    boxrule=0.5mm,         % Espessura da borda
    left=5mm,              % Margem interna à esquerda
    right=5mm,             % Margem interna à direita
    boxsep=5mm,            % Espaçamento entre o conteúdo e a borda
    breakable,             % Permite quebra de página
}
%-- Tabelas
\usepackage{booktabs} % Para criar tabelas com melhor aparência
\usepackage{caption} % Para configurar a legenda da tabela
% Configurações gerais de legendas conforme normas ABNT
\captionsetup{
    labelfont=bf,             % Deixa o rótulo (e.g., Tabela, Figura) em negrito
    textfont=normalfont,      % Texto normal para o título da legenda
    singlelinecheck=false,    % Permite que legendas de uma linha não sejam centralizadas
    labelsep=space,           % Espaço entre o rótulo e o título da legenda
    skip=6pt,
    justification=raggedright % Alinha o texto da legenda à esquerda
}

% Configurações para espaçamento das legendas
\setlength{\abovecaptionskip}{12pt} % Espaço acima da legenda


% Definindo o espaçamento entre as linhas das tabelas (padding vertical)
\renewcommand{\arraystretch}{1.8} % Aumenta o espaçamento entre as linhas das tabelas

% Ajustando o padding horizontal nas células
\setlength{\tabcolsep}{10pt} % Valor padrão é 6pt, ajuste conforme necessário

% Definindo formatação padrão para as colunas das tabelas
\newcolumntype{L}{>{\raggedright\arraybackslash}p{4cm}} % Alinhamento à esquerda com padding, largura 4cm
\newcolumntype{M}{>{\raggedright\arraybackslash}p{11cm}} % Alinhamento à esquerda com padding, largura 11cm

% Configuração do pacote para legendas
\usepackage{caption}
\captionsetup[table]{labelfont=bf, font=small, labelsep=space, justification=centering, skip=0.5\baselineskip}

% Outras configurações gerais
\usepackage{geometry}
\geometry{left=3cm,right=2cm,top=3cm,bottom=2cm}
%---
% FIM
%---