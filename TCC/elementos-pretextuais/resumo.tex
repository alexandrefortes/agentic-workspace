\begin{resumo}
\begin{SingleSpace}
Este trabalho apresenta a aplicação de modelos de linguagem de grande escala (LLMs), especificamente o GPT-4, na automação da avaliação de atendimentos ao cliente, validando sua eficácia por meio de uma prova de conceito (PoC) implementada em um ambiente real na Efí Bank. A hipótese inicial propôs que o LLM poderia reduzir a necessidade de intervenção humana direta na classificação e análise de chamados, permitindo que ajustes fossem realizados por profissionais não especializados em TI, como líderes de atendimento ou analistas de qualidade. Os resultados corroboram essa hipótese, destacando a flexibilidade dos LLMs em se adaptarem a novas demandas com simples ajustes de prompts. Além disso, o projeto demonstrou que a automação pode realizar milhares de análises com um custo operacional competitivo, liberando os profissionais para focarem em atividades de maior valor agregado. Entre os principais desafios enfrentados, destacam-se a sensibilidade dos LLMs ao formato dos prompts e o fenômeno das ``alucinações", em que o modelo pode gerar dados inexistentes ao tentar responder a uma solicitação inadequada. Para mitigar esses desafios, foram implementadas estratégias de refinamento dos prompts e normalização de dados quantitativos.
Os resultados obtidos demonstram que, além de serem economicamente viáveis, os LLMs podem superar as limitações de análises humanas tradicionais em termos de escala e consistência. O projeto também identificou oportunidades de aplicação dessa tecnologia em outras áreas da organização, como na detecção de fraudes e na automação de processos comerciais, ampliando o impacto estratégico da automação cognitiva dentro da empresa.
\end{SingleSpace}
\vspace{\onelineskip}
\textbf{Palavras-chave}: Inteligência artificial generativa. Engenharia de prompt. Modelos de linguagem de grande escala. Automação cognitiva. ChatGPT. Atendimento ao cliente. 
\end{resumo}